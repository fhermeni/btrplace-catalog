% !TEX root =  main.tex
\chapter{Notations}
\label{ch: annexe}
This chapter describes the notations that are used in the catalog to depict configurations or constraints 

\section{Describing a configuration}
\label{sec: config signature}
A configuration depicts the state of servers and virtual machines (see Chapter~\ref{ch: reconfiguration}), and the current placement of virtual machines. For convenience, we propose here a textual, human readable, format to describe a configuration. Listing~\ref{lst: config ebnf} describes the textual format for a configuration using the Extended Backus-Naur Form (EBNF)~\cite{ebnf}.
%
In addition, every identifier in a configuration is supposed to be unique.


\begin{myListing}
\begin{lstlisting}
configuration = server (endl+ server)* (endl+ waitings)? endl*;
on_id = id;
off_id = "(" id ")";
paused_id = "!" id
waitings = "?" ":" on_id+;
server = (server_off | server_on);
server_off = off_id;
server_on = on_id ":" vm*;
vm = on_id | off_id | paused_id;
endl = "\n";
letter = "a" ... "z" | "A" ... "Z";
digit = "0" ... "9";
id = letter (letter | digit)*;
\end{lstlisting}
\caption{EBNF definition of a configuration}\label{lst: config ebnf}
\end{myListing}


Listing~\ref{lst: cfg sample} depicts a sample configuration composed of 3 servers and 5 virtual machines.
Server \cstr{N1} and \cstr{N2} are in the \st{Online} state. \cstr{N1} is hosting 3 virtual machines that are \cstr{VM1}, \cstr{VM2}, and \cstr{VM3}. \cstr{VM1} and \cstr{VM2} are in the \st{Running} state while \cstr{VM3} is in the \st{Suspended} state.
The server \cstr{N2} is hosting the virtual machine \cstr{VM4} that is in the \st{Paused} state.
The server \cstr{N3} is in the \st{Offline} state.
Finally, the virtual machine \cstr{VM6} is in the \st{Waiting} state.

\begin{myListing}
\begin{lstlisting}
N1 : VM1 VM2 (VM3)
N2 : !VM4
(N3)
?  : VM6
\end{lstlisting}
\caption{A sample well-formed configuration.}\label{lst: cfg sample}
\end{myListing}

\section{Describing a constraint}
\label{sec: cstr signature}

\subsection{Declaration}
Each constraint presented in the catalog has a specific signature. In practice, a constraint has
an unique identifier and takes into account a variable amount of parameters.
Listing~ref{let: cstr decl ebnf} describes the textual format for a constraint signature using the EBNF.

\begin{myListing}
\begin{lstlisting}
constraint_decl = id "(" params ");
id = letter (letter | digit | "_")*;
letter = "a" ... "z" | "A" ... "Z";
digit = "0" ... "9";

params = param ("," param) *;
param = id ":" type;
type = (VM_t | server_t | number_t | set_t | string_t);
VM_t = "VM";
server_t = "server";
number_t = "number";
string_t = "string";
set_t = "set<" type ">";
\end{lstlisting}
\caption{EBNF definition of a constraint signature}\label{lst: cstr decl ebnf}
\end{myListing}

The following signature declares a constraint named \cstr{foo}, that takes as parameters a set of VMs named \cstr{s1}, a set of servers named \cstr{s2}, a number \cstr{x}, and a string \cstr{y}:

\smallskip
\cstr{foo(s1:set<VM>, s2:set<set<server>>, x:number, y:string)}

\subsection{Usage}

Constraints deserves to be used to indicate management restrictions. Listing~\ref{lst: cstr usage ebnf} describes the textual format for a constraint call using the EBNF.

\begin{myListing}
\begin{lstlisting}
constraint_ref & id "(" params ");
params & param ("," param)*;
param & id | set | string;
set & { params? };

string & """ (letter | digit)* """;
id & letter (letter | digit | "_")*;
letter & "a" ... "z" | "A" ... "Z";
digit & "0" ... "9";
\end{lstlisting}
\caption{EBNF definition of a constraint signature}\label{lst: cstr usage ebnf}
\end{myListing}

Following example shows a sample usage of the constraint \cstr{foo} declared previously.
The first argument is a set of 3 elements named \cstr{VM1},\cstr{VM2},\cstr{VM3}. From the constraint signature, each of these elements has to be a virtual machine.
The second parameter is a set of 2 servers that are named \cstr{N1}, and \cstr{N2}.
The third parameter is the number 5.
The fourth parameter is the string "bar" :

\smallskip
\cstr{foo(\{VM1, VM2, VM3\}, \{\{N1, N2\},\{N3\}\}, 5, "bar")}

