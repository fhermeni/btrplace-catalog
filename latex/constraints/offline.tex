% !TEX root =  ../main.tex
\section{Offline}

\subsection{Definition}

\subsubsection{Signature} \cstr{offline(s : set<server>)}

The \cstr{offline} constraint forces every server in \cstr{s} to be set in the \st{Offline} state.

\subsubsection{Usage}

This constraint deserves first hardware maintenance concerns. Using this constraint, one datacenter administrator can turn off a set of servers to perform maintenance operation on the hardware

\classification{offline}{datacenter administrator}{Servers state}{Resource management}

\subsubsection{Example}

\fullVersion{
\subsection{Model}

The \cstr{offline} constraint is modeled in terms of the RPs variable by restricting
the state variable of each involved server to 0.

\begin{equation*}
\begin{split}
\forall N \in \mathcal{N},\ offline(N) & \triangleq\\
&	\forall n_i \in N, n_i^q = 0
\end{split}
\end{equation*}

\subsection{Violation detection}

To compute the list of violating elements, it is a necessary to check for each of 
the involved  server that are still online. VMs hosted on that server, including 
VMs in the \st{running}, the \st{paused}, and the \st{sleeping} states, have to be
relocated. This detection is optimal.

\subsection{Availability}

\subsubsection{In {\btrp}}

The constraint is available in {\btrp} under the name \cstr{offline}. It is
modeled by instantiated the state variable of each involved server to 1:

\begin{equation*}
\begin{split}
\forall N \in \mathcal{N},\ offline(N) & \triangleq\\
&	\forall n_i \in N, eq(n_i^q, 0)
\end{split}
\end{equation*}
}

\subsection{See also}
\subsubsection{Related constraints}
\begin{itemize}
\item \cstrref{online}: The opposite constraint that is used to force servers to being set in the \st{Online} state.
\item \cstrref{maxOnlines}:  To restrict the maximum number of servers that are simultaneously in the \st{Online} state.
\end{itemize}