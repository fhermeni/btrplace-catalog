% !TEX root =  ../main.tex
\section{Online}

\subsection{Definition}

\subsubsection{Signature} \cstr{online(s : set<server>)}

The \cstr{online} constraint forces every server in \cstr{s} to be set in the \st{Online} state.

\subsubsection{Usage}

This constraint deserves first the necessity of having servers available to host VMs.
This constraint is also useful in a context where servers can not be managed, \ie turned off.

\classification{online}{datacenter administrator}{Servers state}{Resource management}

\subsubsection{Example}

\fullVersion{
\subsection{Model}

The \cstr{online} constraint is modeled in terms of the RPs variable by restricting
the state variable of each involved server to 1.

\begin{equation*}
\begin{split}
\forall N \in \mathcal{N},\ online(N) & \triangleq\\
&	\forall n_i \in N, n_i^q = 1
\end{split}
\end{equation*}

\subsection{Violation detection}

To compute the list of violating elements, it is a necessary to check for each of 
the involved  server that are still offline. This detection is optimal.

\subsection{Availability}

\subsubsection{In {\btrp}}

The constraint is available in {\btrp} under the name \cstr{offline}. It is
modeled by instantiated the state variable of each involved server to 0:

\begin{equation*}
\begin{split}
\forall N \in \mathcal{N},\ online(N) & \triangleq\\
&	\forall n_i \in N, eq(n_i^q, 1)
\end{split}
\end{equation*}
}

\subsection{See also}
\subsubsection{Related constraints}
\begin{itemize}
\item \cstrref{offline}: The opposite constraint that is used to force servers to being set in the \st{Offline} state.
\end{itemize}

\printListOfInheritance{online}
