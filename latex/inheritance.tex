% !TEX root =  main.tex
\chapter{Constraint Reformulation}
\label{ch: inheritance}

Each VM manager supports a limited set of constraints and a user may  expressed some
constraints that are not available on the VM manager that will host its VMs.

Using the semantic of each constraint, it is possible to establish an inheritance relationship
between constraints. These relations allow to reformulate some constraints that are missing
using a peculiar utilization of more generic constraints while preserving their semantics.
It has however to be noticed that the semantical equivalence between a constraint
and its reformulation  does not ensure the two implementations will have an equivalent 
practical efficiency. Only an analysis of the constraints model and internals may reveal
their relative performances.

This chapter summarizes the inheritance relationship between the constraints presented
in the catalog. First, we present the inheritance graph between all the constraints
to indicate the available reformulations We then present the associated rewriting rules.

\section{Inheritance Graph}

Figure~\ref{fig: inheritance graph} depicts the global inheritance graph between the constraints. 
An arrow between two constraints indicates a possible reformulation.
As an example, a \cstr{gather} constraint can be reformulated using a \cstr{among} constraint.
It is then possible to express the restriction provided by a \cstr{gather} constraint using a
peculiar specialization of a \cstr{among} constraint.
The inheritance relationship is also transitive. A \cstr{gather} constraint can then
be expressed using a \cstr{splitAmong} constraint.
%
A "$+$" vertex indicates a composition of constraints. As an example, the \cstr{quarantine}
constraint can be reformulated using a peculiar composition of \cstr{root} and \cstr{ban} constraints. 

\printInheritanceGraph{fig: inheritance graph}

\section{Rewriting rules}

In this section, we present the rewriting rules associated to each constraint reformulation.
\printListOfInheritances